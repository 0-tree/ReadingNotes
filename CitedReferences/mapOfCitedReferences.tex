\documentclass{article}

%%%%%%%%%%%%%%%%%%%%%%%%%%%%%%%%%%
% style template inspired from Overleaf's arXiv template:
% https://www.overleaf.com/latex/templates/style-and-template-for-preprints-arxiv-bio-arxiv/vknsbpqnxqsk
%%%%%%%%%%%%%%%%%%%%%%%%%%%%%%%%%%

\usepackage{/Users/arthur/Documents/Git/Template_Tex/arxiv_light} % see this for how to import cleanly : https://tex.stackexchange.com/questions/1137/where-do-i-place-my-own-sty-or-cls-files-to-make-them-available-to-all-my-te

\usepackage[utf8]{inputenc} % allow utf-8 input
\usepackage[T1]{fontenc}    % use 8-bit T1 fonts
\usepackage{hyperref}       % hyperlinks
\usepackage{url}            % simple URL typesetting
\usepackage{booktabs}       % professional-quality tables
\usepackage{amsfonts}       % blackboard math symbols
\usepackage{nicefrac}       % compact symbols for 1/2, etc.
\usepackage{microtype}      % microtypography
\usepackage{lipsum}

\usepackage{graphicx}

\title{A Map of Cited References}

\author{Arthur Roullier}

\begin{document}
\maketitle

% \begin{abstract}
%
% \end{abstract}

% keywords can be removed
% \keywords{First keyword \and Second keyword \and More}

In this note I gather together all interesting references cited in articles I have read. Most of the text is copied-pasted from original articles so there may be some syntactic inconsistencies. Original articles (from which the sentence come from) are referenced as [quoted ref]\emph{-[original ref]}.

\section{Machine Learning (not DL)}

\begin{itemize}
	\item ML classics
		\subitem matched filterbank is the optimal classifier for deterministic signals in additive white Gaussian noise Rabiner and Gold [1975]-\cite{balestriero_mad_2018}, ZCA whitening Nam et al. [2014]-\cite{balestriero_mad_2018}, vector quantization Gersho and Gray [2012]-\cite{balestriero_mad_2018}, connection between max-affine splines and K-means (a particular flavor of VQ) has been pointed out in Magnani and Boyd [2009]-\cite{balestriero_mad_2018} in the least-squares regression setting, K-means Jain [2010]-\cite{balestriero_mad_2018}
\end{itemize}


\section{Deep Learning}

\begin{itemize}
	\item relating DNs to splines
		\subitem Montufar et al. [2014]-\cite{balestriero_mad_2018}, Rister and Rubin [2017]-\cite{balestriero_mad_2018}, Unser [2018]-\cite{balestriero_mad_2018}
	\item optimization schemes
		\subitem backpropagation Hecht-Nielsen [1992]-\cite{balestriero_mad_2018}, Gradient Descent (GD) Rumelhart et al. [1988]-\cite{balestriero_mad_2018} being the simplest application of backpropagation, Nesterov Momentum Bengio et al. [2013]-\cite{balestriero_mad_2018} that uses the last performed updates in order to accelerate convergence and finally more complex adaptive methods with internal hyper-parameters updated based on the weights/updates statistics such as Adam Kingma and Ba [2014]-\cite{balestriero_mad_2018}, Adadelta Zeiler [2012]-\cite{balestriero_mad_2018}, Adagrad Duchi et al. [2011]-\cite{balestriero_mad_2018}, RMSprop Tieleman and Hinton [2012]-\cite{balestriero_mad_2018}
	\item DN classics
		\subitem CNN LeCun et al. [1995]-\cite{balestriero_mad_2018}, softmax de Brebisson and Vincent [2015]-\cite{balestriero_mad_2018}, autoencoders Vincent et al. [2008]-\cite{balestriero_mad_2018}, generative adversarial network (GAN) Goodfellow et al. [2014]-\cite{balestriero_mad_2018}, recurrent neural network (RNN) Graves [2013]-\cite{balestriero_mad_2018}, the ability of certain DNs to approximate an arbitrary functional/operator mapping has been well established Cybenko [1989]-\cite{balestriero_mad_2018},  LSTMs and GRUs Cho et al. [2014]-\cite{balestriero_mad_2018}, Chung et al. [2014]-\cite{balestriero_mad_2018}, Graves [2013]-\cite{balestriero_mad_2018}, study of adversarial examples Szegedy et al. [2013]-\cite{balestriero_mad_2018}, batch normalization Ioffe and Szegedy [2015]-\cite{balestriero_mad_2018}, probabilistic models for deep learning Patel et al. [2016]-\cite{balestriero_mad_2018}, dropout Gal and Ghahramani [2016]-\cite{balestriero_mad_2018}
	\item DN news
		\subitem spherical softmax  de Brebisson and Vincent [2015]-\cite{balestriero_mad_2018}, the capability of a shallow DN ($L = 2$) with many partition regions (large $R^{(1)}$) to approximate a complicated operator has been studied numerically in Zhang et al. [2016]-\cite{balestriero_mad_2018}
	\item DN understanding (roughly \emph{five} camps (\cite{balestriero_mad_2018}))
		\subitem probing and measuring networks to visualize their inner workings Zeiler and Fergus [2014]-\cite{balestriero_mad_2018}
		\subitem analyzing their properties such as expressive power Cohen et al. [2016]-\cite{balestriero_mad_2018}, loss surface geometry Lu and Kawaguchi [2017]-\cite{balestriero_mad_2018}, Soudry and Hoffer [2017]-\cite{balestriero_mad_2018}, nuisance management Soatto and Chiuso [2016]-\cite{balestriero_mad_2018}, sparsification Papyan et al. [2017]-\cite{balestriero_mad_2018}, and generalization abilities
		\subitem new mathematical frameworks that share some (but not all) common features with DNs Bruna and Mallat [2013]-\cite{balestriero_mad_2018}
		\subitem probabilistic generative models from which specific DNs can be derived Arora et al. [2013]-\cite{balestriero_mad_2018}, Patel et al. [2016]-\cite{balestriero_mad_2018}
		\subitem information theoretic bounds Tishby and Zaslavsky [2015]-\cite{balestriero_mad_2018}.
	\item generalization
		\subitem the question of which properties of real-world networks explain good generalization behavior has attracted considerable attention [Lan02; LC02; HC93; HS97; Bal+15; Bal+16; Cha+17; Kes+17; DR17; Sch17; Ney+17; NBS18; Aro+18]-\cite{zhou_compressibility_2018} ; see Arora [Aro17]-\cite{zhou_compressibility_2018} for a review of recent advances
	\item compression
		\subitem  See Cheng et al. [Che+18]-\cite{zhou_compressibility_2018} for a survey of recent results in this field. Eliminating unnecessary weights from neural networks (pruning) (LeCun et al., 1990; Hassibi \& Stork, 1993; Han et al., 2015; Li et al., 2016)-\cite{frankle_lottery_2018} can reduce parameter-counts by more than 90\% while maintaining accuracy. Doing so diminishes the size (Han et al., 2015; Hinton et al., 2015)-\cite{frankle_lottery_2018} or energy consumption (Yang et al., 2017; Molchanov et al., 2016; Luo et al., 2017)-\cite{frankle_lottery_2018} of trained networks, making inference more efficient.
\end{itemize}


\section{Approximation Theory}

\begin{itemize}
	\item approximation theory and splines
		\subitem approximation theory Powell [1981]-\cite{balestriero_mad_2018}. spline De Boor et al. [1978]-\cite{balestriero_mad_2018}. A major complication of function approximation with splines in general is the need to jointly optimize both the spline parameters $\alpha, \beta$ and the input domain partition $\Omega$ (the ``knots'' for a 1D spline) Bennett and Botkin [1985]-\cite{balestriero_mad_2018}. However, if a multivariate affine spline is constrained to be globally convex, then it can be rewritten as a max-affine spline Hannah and Dunson [2013]-\cite{balestriero_mad_2018}, Magnani and Boyd [2009]-\cite{balestriero_mad_2018}, meaning that they are adaptive partitioning splines Magnani and Boyd [2009]-\cite{balestriero_mad_2018}.
\end{itemize}





\bibliographystyle{unsrt}
\bibliography{/Users/arthur/Documents/Git/Template_Tex/myLibrary.bib}



\end{document}
