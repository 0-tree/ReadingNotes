\documentclass{article}

%%%%%%%%%%%%%%%%%%%%%%%%%%%%%%%%%%
% style template inspired from Overleaf's arXiv template:
% https://www.overleaf.com/latex/templates/style-and-template-for-preprints-arxiv-bio-arxiv/vknsbpqnxqsk
%%%%%%%%%%%%%%%%%%%%%%%%%%%%%%%%%%

\usepackage{/Users/arthur/Documents/Git/Template_Tex/arxiv_light} % see this for how to import cleanly : https://tex.stackexchange.com/questions/1137/where-do-i-place-my-own-sty-or-cls-files-to-make-them-available-to-all-my-te

\usepackage[utf8]{inputenc} % allow utf-8 input
\usepackage[T1]{fontenc}    % use 8-bit T1 fonts
\usepackage{hyperref}       % hyperlinks
\usepackage{url}            % simple URL typesetting
\usepackage{booktabs}       % professional-quality tables
\usepackage{amsfonts}       % blackboard math symbols
\usepackage{nicefrac}       % compact symbols for 1/2, etc.
\usepackage{microtype}      % microtypography
\usepackage{lipsum}

\usepackage{graphicx}

\title{Reading Notes: \cite{arora_stronger_2018}}

\author{Arthur Roullier}

\begin{document}
\maketitle

% abstract can be removed
% \begin{abstract}
% \lipsum[1]
% \end{abstract}

% keywords can be removed
% \keywords{First keyword \and Second keyword \and More}


\section{Key Quotes}


\subsection{Abstract}
Deep nets generalize well despite having more parameters than the number of training sam- ples. Recent works try to give an explanation using PAC-Bayes and Margin-based analyses, but do not as yet result in sample complexity bounds better than naive parameter counting. The current paper shows generalization bounds that’re orders of magnitude better in practice. These rely upon new succinct reparametrizations of the trained net — a compression that is explicit and efficient. These yield generalization bounds via a simple compression-based framework introduced here. Our results also provide some theoretical justification for widespread empirical success in compressing deep nets.
Analysis of correctness of our compression relies upon some newly identified “noise stability”properties of trained deep nets, which are also experimentally verified. The study of these properties and resulting generalization bounds are also extended to convolutional nets, which had eluded earlier attempts on proving generalization.


\subsection{Concepts}

\begin{itemize}
	\item 
\end{itemize}


\subsection{Results}

\begin{enumerate}
	\item 
\end{enumerate}


\subsection{Thoughts}

\begin{itemize}
	\item 
\end{itemize}


 

\section{Diving In}





\bibliographystyle{unsrt}
\bibliography{/Users/arthur/Documents/Git/Template_Tex/myLibrary.bib}



\end{document}
